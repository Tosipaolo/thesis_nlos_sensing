\chapter{Outlook and Conclusion}

This thesis investigated the feasibility of \gls{isac} in the context of 5G Advanced and 6G wireless systems for \gls{nlos} intrusion detection on factory floors.
The main objective of this work was to explore the potential of \gls{nlos} sensing techniques and their applications in integrated radar systems at mmWave. 

The results obtained from processing real-word measurements suggest that use cases with NLOS conditions should be considered in future studies in ISAC.
With a focus on intrusion detection this work suggests that, with further analysis of the NLOS radar returns and the corresponding processing techniques, NLOS sensing at mmWave is possible.

The main contributions of this work can be summarized as follows:

\begin{itemize}
	\item Real-word NLOS sensing trials and identification of challenges due to communication requirements.
	\item Proposal of \gls{csi} processing approaches for avoiding unwanted replicas in the periodogram.
	\item Definition of requirements for NLOS sensing and validation on measurements from the prototype.
\end{itemize}

This investigations on \gls{nlos} \gls{isac} for intrusion detection use cases only represent an introductory study on the matter.
Follow-up investigations will be required to fully explore new solutions and exploit the capabilities of the system in different scenarios.
