% !TeX spellcheck = <none>
\chapter{Radar detection and thresholding}
\label{chap:radar thresholding}

Detection is commonly defined as the process of analysing the radar data and determining whether it consists of interference only or interference plus echoes from a target of interest \cite{Richards_Scheer_Holm_2010}. Detection is accomplished by setting a threshold for the region of interest, based on the level of interference, and deciding wheter any part of this region is "bright" enough compared to the background.

In many radar systems interference consists not only in receiver noise, but also in clutter components and noise jamming. Interference due to clutter can be quite varying in the region of interest, depending on range, angle and Doppler cells in the vicinity of a target. This limitation makes the use of dynamic thresholding necessary. The idea is to be able to adjust the threshold to the local levels of interference in order to obtain a constant false-alarm rate (CFAR).

%TODO generate image showing fixed vs adaptive threshold

Radar users are typically concerned with determining (or defining) the probability of detecting a target $P_D$ and the probability of false alarm $P_{FA}$.  By utilizing the knowledge of the desired $P_{FA}$, noise statistics, and detector design is possible to decide the appropriate threshold level to be used at the detector's output.

% TODO asses feasibility of confirmation system (does not work for clutter)

One common method for significantly reducing false alarms is to use a confirmation system that requires each target to be detected twice: once in the initial search and then in a subsequent confirmation process.
However, there is still the possibility that a noise spike will remain in the confirmation measure and the target will not. In addition, this approach can suffer from peaks generated by clutter or spectral artefacts that may persist over time. 

% TODO consider putting something about clutter statistics

In this work the following approaches will be considered and their respective advantages and drawbacks analysed.



\section{Constant False Alarm Rate}

In OFDM radar, a \textit{false alarm} occurs when the target detector determines the presence of a target at a range and relative speed that did not contribute to the received matrix $\bm{F}_{Rx}$.

In order to discriminate noise from signal power, the periodogram is subjected to an hypotesis test with an appropriate threshold.

\begin{align}
	\text{Per}_{\bm{F}}(n,m) \quad\mathop{\gtrless}_{H_1}^{H_0}  \quad \eta
\end{align}

Where $H_0$ is the null hypotesis, target not present, and $H_1$ is the hypotesis where the reflection from a target contributes to the received power in the bin uder test.
The probability that any bin of the periodogram exceeds the threshold when only noise power $Z$ is present is:

\begin{align}
	p_{\textbf{FA},bin} = \text{Per}(Z > \eta) = \int_\eta^{\infty} f_z(z|H_0)dz = 1 - F_z(\eta | H_0) = e^{-\dfrac{\eta}{\sigma_n^2}}
\end{align}
 
Where $f_z(z|H_0$ and $F_z(\eta | H_0)$ are the PDF and CDF of the random variable $Z$. The observed noise contribution is given by the magnitude squared of the AWGN with power $\sigma_n^2$.
 
The threshold for a given false alarm rate per bin is obtained as

\begin{align}
	\eta = -\sigma_n^2 \ln(p_{\text{FA},bin}) 
\end{align} 

Optimality for this detection mathod and a more thorough theoretical analysis can be found in chapter 15 of \cite{Richards_Scheer_Holm_2010}.
For the full (non zero padded) periodogram the false alarm probability is

\begin{align}
	p_\text{FA} = 1 - (1 - p_{\text{FA},bin})^{NM}
\end{align}

Solving this for $p_{FA,bin}$ we obtain the value of the threshold as 

\begin{align}
	\eta = \sigma_N^2 \ln{(1 - (1 - p_{\text{FA},bin})^{NM})}
\end{align}

Zero padding the periodogram does not add information useful for the target estimation problem, but it only reduces quantization noise. The contribution of a single bin is spread out between multiple contiguous bins when zero padding.

Noise power is estimated from the periodogram averaging over bins where the target is not expected. For an OFDM signal the two main assumprion are:

\begin{itemize}
	\item the CP duration is larger than the round-trip propagation time for the furthermost target
	\item subcarrier spacing is at least one order of magnitude larger than the largest occurring doppler shift
\end{itemize}

From the first assuption the maximum target delay and consequently the maximum range bin in which we can observe a target are given by
\begin{align}
	\tau_{max} =& \frac{N_{max}}{N_{\text{Per}}}T = T_G \\
	N_{max} =& \frac{T_G}{T}N_{\text{Per}}
\end{align} 

The maximum likelihood estimate for the noise power is found by averaging over one or more rows after $N_{max}$

\begin{align}
	\hat{\sigma}^2 = \frac{1}{M_{\text{Per}}K} \sum_{k=1}^K \sum_{m=1}^{M_{\text{Per}}} \text{Per}_{\bm{F}}(N_{max}+k, m).
\end{align}

\section{Cell-averaging CFAR}
\label{sec:cell averaging CFAR}

The generic CFAR detector described in the previous section defines a threshold, based on noise power, to be applied to the whole periodogram. More advanced algorithms for CFAR detection belong to the family of cell-averaging CFAR (CA-CFAR).

In a noisy environment, very weak echo signals may be lost in the case of a fixed threshold across the periodogram. CA-CFAR estimates the threshold level for each cell-under-test (CUT).
In the one-dimensional case 


