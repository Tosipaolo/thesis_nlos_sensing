% !TeX spellcheck = <none>
\chapter{Radar detection and thresholding}
\label{chap:radar thresholding}

Detection is commonly defined as the process of analysing the radar data and determining whether it consists of interference only or interference plus echoes from a target of interest \cite{Richards_Scheer_Holm_2010}. Detection is accomplished by setting a treshold for the region of interest, based on the level of interference, and deciding wheter any part of this region is "bright" enough compared to the background.

In many radar systems interference consists not only in receiver noise, but also in clutter components and noise jamming. Interference due to clutter can be quite varying in the region of interest, depending on range, angle and Doppler cells in the vicinity of a target. This limitation makes the use of dynamic thresholding necessary. The idea is to be able to adjust the threshold to the local levels of interference in order to obtain a constant false-alarm rate (CFAR).

%TODO generate image showing fixed vs adaptive threshold

The radar user is commonly interested in defining the probability of detecting a target $P_D$ and the probability of false alarm $P_{FA}$. Given the knowledge of the desired $P_{FA}$, noise statistics, and detector design is possible to decide the threshold level to be used at the detector's output.

% TODO asses feasibility of confirmation system (does not work for clutter)

A common way to considerably reduce false alarms is to use a confirmation system, which requires that any target detection occurs twice: once in the initial search and subsequently in a successive confirmation process.
Howerver there is still the possibility that a noise spike persists in the confirmation measure and the target will not. Moreover, this approach can suffer in presence of peaks generated by clutter or spectral artifacts that may persist over time. 

% TODO consider putting something about clutter statistics
